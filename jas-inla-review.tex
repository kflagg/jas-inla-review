% interacttfssample.tex
% v1.05 - August 2017

\documentclass[]{interact}

%\usepackage{epstopdf}% To incorporate .eps illustrations using PDFLaTeX, etc.
\usepackage[caption=false]{subfig}% Support for small, `sub' figures and tables
%\usepackage[nolists,tablesfirst]{endfloat}% To `separate' figures and tables from text if required

%\usepackage[doublespacing]{setspace}% To produce a `double spaced' document if required
%\setlength\parindent{24pt}% To increase paragraph indentation when line spacing is doubled
%\setlength\bibindent{2em}% To increase hanging indent in bibliography when line spacing is doubled

\usepackage[numbers,sort&compress]{natbib}% Citation support using natbib.sty
\bibpunct[, ]{[}{]}{,}{n}{,}{,}% Citation support using natbib.sty
\renewcommand\bibfont{\fontsize{10}{12}\selectfont}% Bibliography support using natbib.sty

%\theoremstyle{plain}% Theorem-like structures provided by amsthm.sty
%\newtheorem{theorem}{Theorem}[section]
%\newtheorem{lemma}[theorem]{Lemma}
%\newtheorem{corollary}[theorem]{Corollary}
%\newtheorem{proposition}[theorem]{Proposition}

%\theoremstyle{definition}
%\newtheorem{definition}[theorem]{Definition}
%\newtheorem{example}[theorem]{Example}

%\theoremstyle{remark}
%\newtheorem{remark}{Remark}
%\newtheorem{notation}{Notation}

\begin{document}

\articletype{REVIEW ARTICLE}% Specify the article type or omit as appropriate

\title{The Integrated Nested Laplace Approximation applied to spatial point process models}

\author{
\name{Kenneth Flagg\thanks{CONTACT Kenneth Flagg. Email: kenneth.flagg@montana.edu} and Andrew Hoegh}
\affil{Montana State University, Bozeman, MT}
}

\maketitle

\begin{abstract}
This template is for authors who are preparing a manuscript for a Taylor \& Francis journal using the \LaTeX\ document preparation system and the \texttt{interact} class file, which is available via selected journals' home pages on the Taylor \& Francis website.
\end{abstract}

\begin{keywords}
\end{keywords}


\section{Introduction}

spatial prediction is a high-dimensional problem

many dependent latent variables but joint dist not necessarily needed

INLA fast approximation for marginals useful for mapping \cite{rueetal}


\section{Log-Gaussian Cox Process}

Poisson process intensity $\lambda(\mathbf{s})$ events per unit area

Model $\log\lambda(\mathbf{s}) = Z(\mathbf{s})$, $Z(\mathbf{s})$ spatial Gaussian process

Random continuous function: $Z(\mathbf{s})$ a Gaussian random variable, Mean $\mu$ Matern covariance function

Poisson process log-likelihood: $\ell(\lambda) = C - \int \lambda(\mathbf{s}) d\mathbf{s} + \sum \log(\lambda(\mathbf{s_{i}}))$

(Need to add covariate to notation)

\section{INLA}

Integrated Nested Laplace Approximation

Bayesian Hierarchical models, many latent Gaussian variables, few parameters

Laplace approximation in general: $\int \exp[h(x)]\mathrm{d}x$, Taylor expansion of $h(x)$

Example from @rinla

- $\mathbf{y} = (y_{1}, \dots, y_{n})'$ independent Gaussian observations

- $y_{i} \sim \mathrm{N}(\theta, \sigma^{2})$

- $\theta \sim \mathrm{N}(\mu_{0}, \sigma_{0}^{2})$

- $\psi = 1/\sigma^{2}$, $\psi \sim \mathrm{Gamma}(a, b)$

- The posterior distribution of $\psi$:
$$p(\psi|\mathbf{y}) \propto \frac{p(\mathbf{y} | \theta, \psi) p(\theta) p(\psi)}
{p(\theta | \psi, \mathbf{y})}$$

- Laplace approximation:
$$\tilde{p}(\psi|\mathbf{y}) \propto \frac{p(\mathbf{y} | \theta, \psi) p(\theta) p(\psi)}
{\tilde{p}_{G}(\theta^{*} | \psi, \mathbf{y})}$$

Repeat for $\theta$, will depend on $\psi$

Provides marginal posteror for one entry at a time of a vector $\boldsymbol{\theta}$


\section{The SPDE Approach}

GP has dense covariance matrix

SPDE result approximates GP with CAR and sparse covariance matrix \cite{lindgrenetal}

$$(\kappa^{2} - \Delta)^{\alpha / 2} (\tau Z(\mathbf{s})) = W(\mathbf{s})$$

Choose nodes $\mathbf{s}_{i}$ to model $Z(\mathbf{s}_{i})$, build a triangular mesh, autoregressive model on the nodes

Change of basis: barycentric coordinates allow sparse matrices and simple linear inerpolation


\section{Going Off the Grid}

Log-likelihood:
$$\ell(Z) = C - \int \exp\left[Z(\mathbf{s})\right] d\mathbf{s} + \sum Z(\mathbf{s_{i}})$$

\cite{simpsonetal}:
$$\ell(Z) \approx C - \sum_{i} \tilde{\alpha}_{i} \exp\left[\sum_{j} z_{j}\phi_{j}(\tilde{\mathbf{s}}_{i})\right] + \sum_{i} \sum_{j} z_{j}\phi_{j}(\mathbf{s_{i}})$$

(Poisson distribution)


\section{Variable Sampling Effort}

Observed a thinned process

Thinning process can be known or unknown

Scale SPDE node integration weights by thinning probabilities when known

Incorporate log-linear model for thinning probability when unknown \cite{yuanetal}


\section{Illustrations}

Examples with data, maybe \texttt{bei} dataset, Victorville, or a simulation study


\section{Conclusion}


%\section*{Acknowledgement(s)}

%An unnumbered section, e.g.\ \verb"\section*{Acknowledgements}", may be used for thanks, etc.\ if required and included \emph{in the non-anonymous version} before any Notes or References.

%\section*{Disclosure statement}

%An unnumbered section, e.g.\ \verb"\section*{Disclosure statement}", may be used to declare any potential conflict of interest and included \emph{in the non-anonymous version} before any Notes or References, after any Acknowledgements and before any Funding information.

%\section*{Funding}

%An unnumbered section, e.g.\ \verb"\section*{Funding}", may be used for grant details, etc.\ if required and included \emph{in the non-anonymous version} before any Notes or References.

%\section*{Notes on contributor(s)}

%An unnumbered section, e.g.\ \verb"\section*{Notes on contributors}", may be included \emph{in the non-anonymous version} if required. A photograph may be added if requested.

%\section*{Nomenclature/Notation}

%An unnumbered section, e.g.\ \verb"\section*{Nomenclature}" (or \verb"\section*{Notation}"), may be included if required, before any Notes or References.

%\section*{Notes}

%An unnumbered `Notes' section may be included before the References (if using the \verb"endnotes" package, use the command \verb"\theendnotes" where the notes are to appear, instead of creating a \verb"\section*").

\bibliographystyle{tfs}
\bibliography{jas-inla-review}

\end{document}
