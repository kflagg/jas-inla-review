% interacttfssample.tex
% v1.05 - August 2017

\documentclass[]{interact}

%\usepackage{epstopdf}% To incorporate .eps illustrations using PDFLaTeX, etc.
\usepackage[caption=false]{subfig}% Support for small, `sub' figures and tables
%\usepackage[nolists,tablesfirst]{endfloat}% To `separate' figures and tables from text if required

%\usepackage[doublespacing]{setspace}% To produce a `double spaced' document if required
%\setlength\parindent{24pt}% To increase paragraph indentation when line spacing is doubled
%\setlength\bibindent{2em}% To increase hanging indent in bibliography when line spacing is doubled

\usepackage[numbers,sort&compress]{natbib}% Citation support using natbib.sty
\bibpunct[, ]{[}{]}{,}{n}{,}{,}% Citation support using natbib.sty
\renewcommand\bibfont{\fontsize{10}{12}\selectfont}% Bibliography support using natbib.sty

%\theoremstyle{plain}% Theorem-like structures provided by amsthm.sty
%\newtheorem{theorem}{Theorem}[section]
%\newtheorem{lemma}[theorem]{Lemma}
%\newtheorem{corollary}[theorem]{Corollary}
%\newtheorem{proposition}[theorem]{Proposition}

%\theoremstyle{definition}
%\newtheorem{definition}[theorem]{Definition}
%\newtheorem{example}[theorem]{Example}

%\theoremstyle{remark}
%\newtheorem{remark}{Remark}
%\newtheorem{notation}{Notation}

\begin{document}

\articletype{REVIEW ARTICLE}% Specify the article type or omit as appropriate

\title{The Integrated Nested Laplace Approximation applied to spatial point process models}

\author{
\name{Kenneth Flagg\thanks{CONTACT Kenneth Flagg. Email: kenneth.flagg@montana.edu} and Andrew Hoegh}
\affil{Montana State University, Bozeman, MT}
}

\maketitle

\begin{abstract}
This template is for authors who are preparing a manuscript for a Taylor \& Francis journal using the \LaTeX\ document preparation system and the \texttt{interact} class file, which is available via selected journals' home pages on the Taylor \& Francis website.
\end{abstract}

\begin{keywords}
\end{keywords}


\section{Introduction}

\cite{rueetal}

\section{Log-Gaussian Cox Process}

\section{INLA}

\section{The SPDE Approach}

\section{Going Off-Grid}

\section{Illustrations}

\section{Conclusion}


%\section*{Acknowledgement(s)}

%An unnumbered section, e.g.\ \verb"\section*{Acknowledgements}", may be used for thanks, etc.\ if required and included \emph{in the non-anonymous version} before any Notes or References.

%\section*{Disclosure statement}

%An unnumbered section, e.g.\ \verb"\section*{Disclosure statement}", may be used to declare any potential conflict of interest and included \emph{in the non-anonymous version} before any Notes or References, after any Acknowledgements and before any Funding information.

%\section*{Funding}

%An unnumbered section, e.g.\ \verb"\section*{Funding}", may be used for grant details, etc.\ if required and included \emph{in the non-anonymous version} before any Notes or References.

%\section*{Notes on contributor(s)}

%An unnumbered section, e.g.\ \verb"\section*{Notes on contributors}", may be included \emph{in the non-anonymous version} if required. A photograph may be added if requested.

%\section*{Nomenclature/Notation}

%An unnumbered section, e.g.\ \verb"\section*{Nomenclature}" (or \verb"\section*{Notation}"), may be included if required, before any Notes or References.

%\section*{Notes}

%An unnumbered `Notes' section may be included before the References (if using the \verb"endnotes" package, use the command \verb"\theendnotes" where the notes are to appear, instead of creating a \verb"\section*").

\section{References}

\bibliographystyle{tfs}
\bibliography{jas-inla-review}

\end{document}
