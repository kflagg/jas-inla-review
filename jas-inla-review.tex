% interacttfssample.tex
% v1.05 - August 2017

\documentclass[]{interact}

%\usepackage{epstopdf}% To incorporate .eps illustrations using PDFLaTeX, etc.
\usepackage[caption=false]{subfig}% Support for small, `sub' figures and tables
%\usepackage[nolists,tablesfirst]{endfloat}% To `separate' figures and tables from text if required

%\usepackage[doublespacing]{setspace}% To produce a `double spaced' document if required
%\setlength\parindent{24pt}% To increase paragraph indentation when line spacing is doubled
%\setlength\bibindent{2em}% To increase hanging indent in bibliography when line spacing is doubled

\usepackage[numbers,sort&compress]{natbib}% Citation support using natbib.sty
\bibpunct[, ]{[}{]}{,}{n}{,}{,}% Citation support using natbib.sty
\renewcommand\bibfont{\fontsize{10}{12}\selectfont}% Bibliography support using natbib.sty

%\theoremstyle{plain}% Theorem-like structures provided by amsthm.sty
%\newtheorem{theorem}{Theorem}[section]
%\newtheorem{lemma}[theorem]{Lemma}
%\newtheorem{corollary}[theorem]{Corollary}
%\newtheorem{proposition}[theorem]{Proposition}

%\theoremstyle{definition}
%\newtheorem{definition}[theorem]{Definition}
%\newtheorem{example}[theorem]{Example}

%\theoremstyle{remark}
%\newtheorem{remark}{Remark}
%\newtheorem{notation}{Notation}

\begin{document}

{\Large\bf Progress this week}

{\large\bf\today}

\begin{itemize}

\item Added some keywords.

\item Turned introduction into complete sentences.

\end{itemize}

\vfill

\textbf{Previous word count:} --- \hfill \textbf{Current word count:}

\textbf{Previous page count:} --- \hfill \textbf{Current page count:}

\pagebreak

\articletype{REVIEW ARTICLE}% Specify the article type or omit as appropriate

\title{The Integrated Nested Laplace Approximation applied to Spatial Point Process Models}

\author{
\name{Kenneth Flagg\thanks{CONTACT Kenneth Flagg. Email: kenneth.flagg@montana.edu} and Andrew Hoegh}
\affil{Montana State University, Bozeman, MT}
}

\maketitle

\begin{abstract}
This template is for authors who are preparing a manuscript for a Taylor \& Francis journal using the \LaTeX\ document preparation system and the \texttt{interact} class file, which is available via selected journals' home pages on the Taylor \& Francis website.
\end{abstract}

\begin{keywords}
INLA, spatial prediction, log-Gaussian Cox process, spatial point process
\end{keywords}


\section{Introduction}
%  {\bf Introduction:} Please introduce one or more current statistical research methods that have been widely used by other discipline(s) in the Introduction section or divide the discussion into two or more subsections within this Introduction section.

Spatial prediction is a high-dimensional inference problem. When the goal of
statistical modeling is to produce a graphical map of a random variable over
space, the model ultimately must be able to predict that random variable at
every pixel of the image. A map image will typically be at least several
hundred by several hundred pixels, so in total there can easily be hundreds of
thoudands of pixels requiring predictions. Thus, even when a model has only
half a dozen parameters, it may include hundreds of thousands of latent
variables.

Spatial point process models further complicate the situation with difficult
likelihoods. {\it (Cite some computational papers --- Baddely?)} Both maximum
likelihood and Bayesian model fitting require integrating the intensity
function over space, but the integral is generally not available in closed
form. Many methods have been introduced including quadrature-based
approximations {\it (cite Baddeley)}, pseudodata approaches
{\it (cite Baddeley/Berman/Turner etc)}, and Markov chain Monte
Carlo~\cite{moellerwaagepetersen}.

Development of the integrated nested Laplace approximation (INLA) has made
accurate approximate model fitting considerably more feasible for a particular
class of log-Gaussian Cox process (LGCP) models. INLA was developed to fit
Bayesian hierarchical models with many latent Gaussian
variables~\cite{rueetal}. A key part of INLA's computational simplicity is that
it calculates the posterior distribution of each latent Gaussian variable one
at a time; that is, it provides only the posterior marginal distributions
rather that the full joint distribution.

When using a LGCP for spatial mapping, two aspects make INLA a suitable
approach. First, the LGCP is driven by a spatial Gaussian process, so the
latent variables are Gaussian. Second, even though the latent variables are
expected to exhibit spatial dependence, their full joint distribution is not
needed. In most situations it suffices to map their predicted values, variance,
and upper and lower interval bounds pointwise across space.

\subsection{Log-Gaussian Cox Process}

Poisson process intensity $\lambda(\mathbf{s})$ events per unit area

Model $\log\lambda(\mathbf{s}) = Z(\mathbf{s})$, $Z(\mathbf{s})$ spatial Gaussian process

Random continuous function: $Z(\mathbf{s})$ a Gaussian random variable, Mean $\mu$ Matern covariance function

Poisson process log-likelihood: $\ell(\lambda) = C - \int \lambda(\mathbf{s}) d\mathbf{s} + \sum \log(\lambda(\mathbf{s_{i}}))$

(Need to add covariates to notation)

\subsection{INLA}
INLA fast approximation for marginals useful for mapping \cite{rueetal}

Integrated Nested Laplace Approximation

Bayesian Hierarchical models, many latent Gaussian variables, few parameters

Laplace approximation in general: $\int \exp[h(x)]\mathrm{d}x$, Taylor expansion of $h(x)$

Example from \cite{rinla}

- $\mathbf{y} = (y_{1}, \dots, y_{n})'$ independent Gaussian observations

- $y_{i} \sim \mathrm{N}(\theta, \sigma^{2})$

- $\theta \sim \mathrm{N}(\mu_{0}, \sigma_{0}^{2})$

- $\psi = 1/\sigma^{2}$, $\psi \sim \mathrm{Gamma}(a, b)$

- The posterior distribution of $\psi$:
$$p(\psi|\mathbf{y}) \propto \frac{p(\mathbf{y} | \theta, \psi) p(\theta) p(\psi)}
{p(\theta | \psi, \mathbf{y})}$$

- Laplace approximation:
$$\tilde{p}(\psi|\mathbf{y}) \propto \frac{p(\mathbf{y} | \theta, \psi) p(\theta) p(\psi)}
{\tilde{p}_{G}(\theta^{*} | \psi, \mathbf{y})}$$

Repeat for $\theta$, will depend on $\psi$

Provides marginal posteror for one entry at a time of a vector $\boldsymbol{\theta}$



\section{Methodology}
%  {\bf Methodology or Approach:} Present the method(s) and relevant applications being reviewed in this or more sections.

\subsection{The SPDE Approach}

GP has dense covariance matrix

SPDE result approximates GP with CAR and sparse covariance matrix \cite{lindgrenetal}

$$(\kappa^{2} - \Delta)^{\alpha / 2} (\tau Z(\mathbf{s})) = W(\mathbf{s})$$

Choose nodes $\mathbf{s}_{i}$ to model $Z(\mathbf{s}_{i})$, build a triangular mesh, autoregressive model on the nodes

Change of basis: barycentric coordinates allow sparse matrices and simple linear inerpolation


\subsection{Going Off the Grid}

Log-likelihood:
$$\ell(Z) = C - \int \exp\left[Z(\mathbf{s})\right] d\mathbf{s} + \sum Z(\mathbf{s_{i}})$$

\cite{simpsonetal}:
$$\ell(Z) \approx C - \sum_{i} \tilde{\alpha}_{i} \exp\left[\sum_{j} z_{j}\phi_{j}(\tilde{\mathbf{s}}_{i})\right] + \sum_{i} \sum_{j} z_{j}\phi_{j}(\mathbf{s_{i}})$$

(Poisson distribution)


\subsection{Variable Sampling Effort -- Delete this subsection?}

Observed a thinned process

Thinning process can be known or unknown

Scale SPDE node integration weights by thinning probabilities when known

Incorporate log-linear model for thinning probability when unknown \cite{yuanetal}


\section{Applications}
%  {\bf Application of your Methodology:} Present some applications of the reviewed methods to some recent data in this or more sections. Also note: Simulation studies can be presented as a sub-section in this section or in a new standalone section.

\subsection{Simulation Study}

\subsection{Data Application}

Examples with data, maybe \texttt{bei} dataset or Victorville


\section{Conclusion and Discussion}
%  {\bf Conclusion and Discussion:} In this conclusion and discussion section, provide conclusions of your review and provide anticipated future direction(s) in this area.


%\section*{Acknowledgement(s)}

%An unnumbered section, e.g.\ \verb"\section*{Acknowledgements}", may be used for thanks, etc.\ if required and included \emph{in the non-anonymous version} before any Notes or References.

%\section*{Disclosure statement}

%An unnumbered section, e.g.\ \verb"\section*{Disclosure statement}", may be used to declare any potential conflict of interest and included \emph{in the non-anonymous version} before any Notes or References, after any Acknowledgements and before any Funding information.

%\section*{Funding}

%An unnumbered section, e.g.\ \verb"\section*{Funding}", may be used for grant details, etc.\ if required and included \emph{in the non-anonymous version} before any Notes or References.

%\section*{Notes on contributor(s)}

%An unnumbered section, e.g.\ \verb"\section*{Notes on contributors}", may be included \emph{in the non-anonymous version} if required. A photograph may be added if requested.

%\section*{Nomenclature/Notation}

%An unnumbered section, e.g.\ \verb"\section*{Nomenclature}" (or \verb"\section*{Notation}"), may be included if required, before any Notes or References.

%\section*{Notes}

%An unnumbered `Notes' section may be included before the References (if using the \verb"endnotes" package, use the command \verb"\theendnotes" where the notes are to appear, instead of creating a \verb"\section*").

\bibliographystyle{tfs}
\bibliography{jas-inla-review}

\end{document}
